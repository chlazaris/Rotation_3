%%%%%%%%%%%%%%%%%%%%%%%%%%%%%%%%%%%%%%%%%
% Daily Laboratory Book
% LaTeX Template
%
% This template has been downloaded from:
% http://www.latextemplates.com
%
% Original author:
% Frank Kuster (http://www.ctan.org/tex-archive/macros/latex/contrib/labbook/)
%
% Important note:
% This template requires the labbook.cls file to be in the same directory as the
% .tex file. The labbook.cls file provides the necessary structure to create the
% lab book.
%
% The \lipsum[#] commands throughout this template generate dummy text
% to fill the template out. These commands should all be removed when 
% writing lab book content.
%
% HOW TO USE THIS TEMPLATE 
% Each day in the lab consists of three main things:
%
% 1. LABDAY: The first thing to put is the \labday{} command with a date in 
% curly brackets, this will make a new page and put the date in big letters 
% at the top.
%
% 2. EXPERIMENT: Next you need to specify what experiment(s) you are 
% working on with an \experiment{} command with the experiment shorthand 
% in the curly brackets. The experiment shorthand is defined in the 
% 'DEFINITION OF EXPERIMENTS' section below, this means you can 
% say \experiment{pcr} and the actual text written to the PDF will be what 
% you set the 'pcr' experiment to be. If the experiment is a one off, you can 
% just write it in the bracket without creating a shorthand. Note: if you don't 
% want to have an experiment, just leave this out and it won't be printed.
%
% 3. CONTENT: Following the experiment is the content, i.e. what progress 
% you made on the experiment that day.
%
%%%%%%%%%%%%%%%%%%%%%%%%%%%%%%%%%%%%%%%%%

%----------------------------------------------------------------------------------------
%	PACKAGES AND OTHER DOCUMENT CONFIGURATIONS
%----------------------------------------------------------------------------------------

\documentclass[idxtotoc,hyperref,openany]{labbook} % 'openany' here removes the gap page between days, erase it to restore this gap; 'oneside' can also be added to remove the shift that odd pages have to the right for easier reading

\usepackage[ 
  backref=page,
  pdfpagelabels=true,
  plainpages=false,
  colorlinks=true,
  bookmarks=true,
  pdfview=FitB]{hyperref} % Required for the hyperlinks within the PDF
  
\usepackage{booktabs} % Required for the top and bottom rules in the table
\usepackage{float} % Required for specifying the exact location of a figure or table
\usepackage{graphicx} % Required for including images
\usepackage{lipsum} % Used for inserting dummy 'Lorem ipsum' text into the template
% Use the listings package in order to be able to insert code:
\usepackage{listings}

\newcommand{\HRule}{\rule{\linewidth}{0.5mm}} % Command to make the lines in the title page
\setlength\parindent{0pt} % Removes all indentation from paragraphs

%----------------------------------------------------------------------------------------
%	DEFINITION OF EXPERIMENTS
%----------------------------------------------------------------------------------------

\newexperiment{example}{This is an example experiment}
\newexperiment{example2}{This is another example experiment}
\newexperiment{example3}{This is yet another example experiment}
\newexperiment{table}{This shows a sample table}
%\newexperiment{shorthand}{Description of the experiment}

%---------------------------------------------------------------------------------------

\begin{document}

%----------------------------------------------------------------------------------------
%	TITLE PAGE
%----------------------------------------------------------------------------------------

\frontmatter % Use Roman numerals for page numbers
\title{
\begin{center}
\HRule \\[0.4cm]
{\Huge \bfseries Laboratory Journal \\[0.5cm] \Large Rotation 3}\\[0.4cm] % Degree
\HRule \\[1.5cm]
\end{center}
}
\author{\Huge Harris A. Lazaris \\ \\ \LARGE lazaris@nyu.edu \\[2cm]} % Your name and email address
\date{Beginning 10 March 2014} % Beginning date
\maketitle

\tableofcontents

\mainmatter % Use Arabic numerals for page numbers

%----------------------------------------------------------------------------------------
%	LAB BOOK CONTENTS
%----------------------------------------------------------------------------------------

% Blank template to use for new days:

%\labday{Day, Date Month Year}

%\experiment{}

%Text

%-----------------------------------------

%\experiment{}

%Text

%----------------------------------------------------------------------------------------

% \labday{Friday, 25 November 2011}

% \experiment{example}

% \lipsum[1]

%-----------------------------------------

% \experiment{example2} % Multiple experiments can be included in a single day, this allows you to segment what was done each day into separate categories

% \begin{figure}[H] % Example of including images
% \begin{center}
% \includegraphics[width=0.5\linewidth]{example_figure}
% \end{center}
% \caption{Example figure.}
% \label{fig:example_figure}
% \end{figure}

%-----------------------------------------

% \experiment{example3}

% \lipsum[3-5]

%----------------------------------------------------------------------------------------

% \labday{Friday, 26 March 2010}

% \experiment{table}

% \begin{table}[H]
% \begin{tabular}{l l l}
% \toprule
% \textbf{Groups} & \textbf{Treatment X} & \textbf{Treatment Y} \\
% \toprule
% 1 & 0.2 & 0.8\\
% 2 & 0.17 & 0.7\\
% 3 & 0.24 & 0.75\\
% 4 & 0.68 & 0.3\\
% \bottomrule
% \end{tabular}
% \caption{The effects of treatments X and Y on the four groups studied.}
% \label{tab:treatments_xy}
% \end{table}

% Table \ref{tab:treatments_xy} shows that groups 1-3 reacted similarly to the two treatments but group 4 showed a reversed reaction.

%----------------------------------------------------------------------------------------

\labday{Monday, 10 March 2014}

\experiment{Collect data concerning the Ren ``Topological domains'' paper.} % You don't need to make a \newexperiment if you only plan on referencing it once

These are the tasks of the day:

\begin{itemize}
\item Go to GSE35156 on GSE and try to download the matrix
      data
\item If you do not find any data there, contact the authors.
\end{itemize}

The data that we are looking for are the matrices that give all the 
Hi-C interactions for the cell types that the Ren group used in the study:
\texttt{Dixon, J. R. et al. Topological domains in mammalian genomes identified by analysis of chromatin interactions. Nature 485, 376–380 (2012)}.
I went on the website \url{http://www.ncbi.nlm.nih.gov/geo/query/acc.cgi?acc=GSE35156} and found that the matrix files listed there are actually
incomplete as they contain only the headers and not the matrix data. I also
did a character count to confirm that the size of the file corresponds
to the headers only and there is not anything else in there, that I missed.

Having proven that the matrix data are not there I e-mailed Bing Ren (biren@ucsd.edu) and the first author of the paper Jesse R. Dixon (j1dixon@ucsd.edu) and requested the processed data concerning the matrices and also 
the topological domains. Moreover, I requested detailed description of the
methods used (trimming of the data, binning etc). I cc'ed the e-mail to
Aris (Tsirigos) as well.



%-----------------------------------------


\labday{Friday, 14 March 2014}

\experiment{Try to calculate Spearman correlation coefficients using HiCNorm} % You don't need to make a \newexperiment if you only plan on referencing it once

These are the tasks of the day:

\begin{itemize}
\item Download the data from ~\url{http://www.people.fas.harvard.edu/~junliu/HiCNorm}
\item Play with the data to see if you can calculate any coefficient.
\item Meet with Aris at 2pm to run the MATLAB code. Anju said that he will be available
early next week. So, we can start on our own.
\item Discuss with Aris the requirements of the two proposals:
	\begin{description}
		\item [HPC Project] Bash scripting, high-performance computing, parallel jobs? What should I mention on the proposal? He wants us to write a brief discussion on the question, its importance, our approach and how we are going to implement it.
		\item [Methods Project] We have to start thinking about it too. The code should be placed on ~\texttt{git}. Larry said that it would be great to use some Python 
		(Scipy, Numpy etc). Also he mentioned that we may need to create a database. What kind of database would be appropriate for our data? 
	\end{description}
\end{itemize}

% \experiment{example}

% \lipsum[6]

%-----------------------------------------

\labday{Monday, 17 March 2014}

\experiment{Run MATLAB code with Aris and Anju for HiC evaluation} % You don't need to make a \newexperiment if you only plan on referencing it once

These are the tasks of the day:

\begin{itemize}
\item Use rep1.mat and rep2.mat as input files for MATLAB.
\item Meet with Aris and Anju (Skype) at 2pm to run the MATLAB code. 
\item Discuss with Aris the requirements of the two proposals:
  \begin{description}
     \item Discuss with Aris the content of the lab meeting 
     presentation to start working on it.
     \item Check for R debugger.
  \end{description}
\end{itemize}

Useful command to extract the chromosomes only: 

\begin{lstlisting}[language=bash]
	more .reg+ | cut -f2 | cut -d' '-f1 | sed 's/^chr//' |\
	sed \'s/X/23/\' > ...
\end{lstlisting}

The presentation is going to be on the Hi--C method in general (Focus on
the Lieberman paper (2009)). Then, the second half will be on the latest 
Ren paper (Nature 2013). Start working on the presentation today.

The data are available in a centralized location and I have created a soft
linked to this location (Dekker-Science-2009). In the directory \texttt{run\_dat\_files} the script that I wrote today is placed:~\texttt{hic\_chr\_by\_chr\_matrix\_comp.R}. Moreover the \texttt{chr\_file} which is also in this directory, provides the names
for the chromosomes (based on 4096KB resolution).

I ran the command:

\begin{lstlisting}[language=bash]
	Rscript hic_chr_by_chr_matrix_comp.R 
	../Dekker-Science-2009/hiclib/hiclib.GM-rep1-HindIII-HiC/
	heatmap-res-4096kb.filtered.dat
	../Dekker-Science-2009/hiclib/hiclib.GM-rep5-NcoI-HiC/
	heatmap-res-4096kb.filtered.dat chr_file spearman false
\end{lstlisting}

to calculate the Hi--C correlation between the first replicate
with HindIII and the replicate 1 with NcoI (for resolution 4096KB):

In the aforementioned example the Spearman correlation is: 0.873 (rounded
to three decimal places). 

\subsubsection*{Future plans}

\begin{enumerate}
  \item Make the Rscript output chromosomes and correlation
  (23 chromosomes and the corresponding correlations) and save
  to file with name determined by the input files.
  \item Write an R script that will use the correlations
  from these files to create boxplots (all boxplots should
  appear on the same plot).
  \item Add the electronic lab-book and the scripts to a private
  repository on GitHub (you may share with Aris).
  \item Ask the guys again for the non-normalized data
\end{enumerate}


\labday{Monday, 24 March 2014}

\experiment{Discuss with Anju how to create the pipeline for the correction of filtered.dat files in order to get the corrected.dat}

These are the tasks of the day:

\begin{enumerate}
	\item Discuss with Anju about the method for Hi--C correction.
	\item Correct the diagram for hiclib 4096kb resolution
	\item Update the lab book with all the images that you have already generated from the analysis of hiclib data (namely 128kb, 1024kb and 4096kb).
	\item Follow Anju's instructions
\end{enumerate}

Anju's instructions:

\begin{enumerate}
	\item Start with the matrix (.dat file)
	\item Load the matrix to memory
	\item Run HiCHarvard.m from MATLAB (ask Aris for the files
	that HiCHarvard requires (the chrom file, the length file and the GC file).
	\item You should get back P which is the normalized (with HiCNorm) version of the matrix that was provided. 
\end{enumerate}
    
% \experiment{example2}

% \lipsum[7]



%---------------------------------------------------------------------------------------
%	FORMULAE AND MEDIA RECIPES
%----------------------------------------------------------------------------------------

% \labday{} % We don't want a date here so we make the labday blank

% \begin{center}
% \HRule \\[0.4cm]
% {\huge \textbf{Formulae and Media Recipes}}\\[0.4cm] % Heading
% \HRule \\[1.5cm]
% \end{center}

%----------------------------------------------------------------------------------------
%	MEDIA RECIPES
%----------------------------------------------------------------------------------------

% \newpage

% \huge \textbf{Media} \\ \\

% \normalsize \textbf{Media 1}\\
% \begin{table}[H]
% \begin{tabular}{l l l}
% \toprule
% \textbf{Compound} & \textbf{1L} & \textbf{0.5L}\\
% \toprule
% Compound 1 & 10g & 5g\\
% Compound 2 & 20g & 10g\\
% \bottomrule
% \end{tabular}
% \caption{Ingredients in Media 1.}
% \label{tab:med1}
% \end{table}

%-----------------------------------------

%\textbf{Media 2}\\ \\

%Description

%----------------------------------------------------------------------------------------
%	FORMULAE
%----------------------------------------------------------------------------------------

% \newpage

% \huge \textbf{Formulae} \\ \\

% \normalsize \textbf{Formula 1 - Pythagorean theorem}\\ \\
% $a^2 + b^2 = c^2$\\ \\

%-----------------------------------------

%\textbf{Formula X - Description}\\ \\

%Formula

%----------------------------------------------------------------------------------------

\end{document}